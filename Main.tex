\documentclass[a4paper,10pt]{article}

\usepackage[utf8]{inputenc}
\usepackage[french]{babel}
\usepackage{microtype}
\usepackage{amsmath,amsthm,amsfonts,amssymb}
\usepackage{graphicx}
\usepackage{lmodern}
\usepackage{amsmath,amsfonts,mathrsfs}
\usepackage{gensymb}
\usepackage{lettrine}
\usepackage{calligra}
\usepackage{tikz}
\usepackage{tikz-cd}
\usepackage{tikz-3dplot}
\usepackage{cancel}
\usepackage{boxedminipage}
\usepackage{mathtools}
\usepackage{esint}
\usepackage{stmaryrd}
\usepackage{minted}
\usepackage[european resistor, european voltage, european current]{circuitikz}
\usepackage{bussproofs}
\usepackage{yfonts}
\usepackage{turnstile}
\usepackage{dsfont}
\usepackage{hyperref}


\begin{document}

\maketitle

\center \title{\large \textbf{Algèbre commutative et théorème des zéros de Hilbert}}
\flushleft

\section{Radical}

Soit $A$ un anneau commutatif.


\flushleft\underline{Definition} -Soit $S\subset A $
\\On appelle radical de $S$ l'ensemble :

\centering $ \sqrt{S}:=\{ a\in A \; | \exists n \in \mathbb{N} \; a^n \in I \} 




\flushleft\underline{Propriété:}Soit $I$ un idéal de $A$. Alors $\sqrt{I}$ est un idéal de $A$ contenant $S$.

\underline{Preuve}:

\item[•] $0 \; \in \sqrt{I}$ car $ 0 \in \ I$

Soient $x$ , $y$ $ \in \sqrt{I}$. Il existe $n$ et $k$ $\in \mathbb{N}$ tels que $x^n \in I$ et $y^k \in I $. 
\item[•] $(x-y)^{n+k}= \sum_{j=0}^{n+k} \binom{n+k}{j}x^j(-y)^{n+k-j}$. Chacun des termes de la somme est dans $I$ donc $ (x-y)^{n+k} \; \in I$.

\item[•] $(xy)^{nk}=(x^n)^k((y)^k)^n \; \in I$ 

Conclusion: $\sqrt{I}$ est un idéal de $I$.

\vspace{1\baselineskip}

\\ \underline{Definition}: Un idéal radiciel est un idéal d’un anneau qui est son propre radical.

\vspace{1\baselineskip}

\underline{Propriété}:

Le radical d’un idéal d’un anneau est radiciel

\underline{Preuve}:

Soit $I$ un idéal.
\item[•] , $\sqrt{I} \subset \sqrt{\sqrt{I}} $ est claire
\item[•]Reciproquement, soit $x \in A $ et $n \in \mathbb{N}$ tel que $x^n \in \sqrt{I} $. Soit maintenant $k \in \mathbb{N} $ tel que $(x^n)^k \in I $. On a donc que $x^{nk} \in I}$, donc $x \in \sqrt{I}$. $\sqrt{\sqrt{I}} \subset \sqrt{I} $

Conclusion: $\sqrt{\sqrt{I}} = \sqrt{I} $


\section{Eléments entiers}

Soit  A  et B deux anneaux commutatifs unitaires et $ f & : & A & \to & B $ un morphisme d'anneaux.

\underline{Défintion} - Soit $ x$ $\in$ B. On dit que $x$ est entier sur A s'il existe $ n\in \mathbb{N} $ et $ a_{1}, a_{2}, ... ,a_{n} $\in$ A  tel que :

\centering $ x^n + f(a_{1})x^{n-1}+ ... + f(a_{n-1})x + f(a_{n}) = 0$

\flushleft On appelle cette éqauation une \textit{relation de dépendance intégrale}

\underline{Propriété} -
Soit $x \in$ B  entier sur A si et seulement si il existe  R sous-anneau de B qui contient A[$x$] et qui soit de type fini comme A-module.

\underline{Démonstration} -

\item[•] Soit $x\in$ B entier dans A. Alors il est clair que l'algèbre A[$x$]  est  engendrée par $x$ dans B et c'est également un  A-module de type fini car engendré par $1,x,...,x^{n-1}$. Donc ici on pose R = A

\item[•] Soit $R$ sous-anneau de $B$ engendré par $r_{1}, ... , r_{m} $ donc $xR \subset R$ donc $\forall  i \in [1,m]$ $\exists\  m_{i,j} \in$ A , $\forall \ j \in [1,m]$ tel que 

\centering $ xr_{i} = \sum_{j=0}^{m} m_{i,j} r_{j} $

\flushleft Notons $  M= (m_{i,j})_{1\le i,j \le m}$. Alors \begin{pmatrix}
r_{1} \\
r_{2} \\
. \\
r_{m} \\
\end{pmatrix}
\in  Ker($$ xI_{m}-M$). Donc on a que si on note C la transposée de la comatrice de $xI_{m}-M$ alors $ C(xI_{m}-M)$ = det($xI_{m}-M$)$I_{m} $ . Donc  det($xI_{m}-M$)$I_{m}$ \times \begin{pmatrix}
r_{1} \\
r_{2} \\
. \\
r_{m} \\
\end{pmatrix}$= 0$.\\Cela implique que  $\forall\ i \in [1,m]$ det($xI_{m}-M)r_{i} = 0$ or $1 \in$ R donc det($xI_{m}-M$)$ = 0$. 
\\En sachant que det($xI_{m}-M$)} est un polymome en $x$, alors il est entier sur A.

Si  $\forall x \in $B , $x$ est entier sur A  alors on dit que B est entier sur A ou que B est une algèbre entière sur A ou encore que $f$ est entier sur A.
L'ensemble des éléments de B entiers sur A est appellée la \textit{cloture intégrale de A dans B} 

\underline{Lemme} - Soit A et B deux anneaux commutatifs unitaires intègres Si A $\subset$ B et que B est entier sur A alors A est un corps si et seulement si B est un corps.

\underline{Démonstration} - 

\item[•] On suppose que A est un corps. Car B est entier sur A :
\\$\forall\ b \in $ \exists \ a_{1}, a_{2}, ... ,a_{n} $\in$ A tel que $ b^n + a_{1}b^{n-1}+ ... + a_{n-1}b + a_{n} = 0$ \\Donc $b(b^{n-1} +a_{1}b^{n-2}+ ... + a_{n-1}) = - a_{n} $ donc $(-a_{n}^{-1}(b^{n-1} +a_{1}b^{n-2}+ ... + a_{n-1}))b = 1} $. Donc b est inversible et car B est intègre c'est un corps.

\item[•]Si B est un corps car A $\subset $ B, soit $a \in $ A alors   $a \in $ B donc $a^{-1} \in$ B  on note $a^{-1}= b $ .
\\Alors $ \exists \ a_{1}, a_{2}, ... ,a_{n}\in$ A tel que $ b^n + a_{1}b^{n-1}+ ... + a_{n-1}b + a_{n} = 0$ donc $b^n =- (a_{1}b^{n-1}+ ... + a_{n-1}b + a_{n}) $ par le même raisonnement on obtient qu'en multipliant par $ a^{n-1}$ on obtient que $a^{-1} = - (a_{1}}+ ... + a_{n-1}a^{n-2} + a_{n}a^{n-1})$ . cela implique que $a^{-1} \in $ A et donc c'est un corps. 

\subsection{Bonus sur les extensions de corps}

\underline{Définitions} - \item[•] Soit k corps commutatif, K est une extension de corps de k si k $\subset$ K et que k est un sous corps de K.
\\ \textbf{Exemple :} $\mathbb{C} $ est une extension de corps sur le corps des réels $\mathbb{R}$$
\\ \item[•] On dit que l'extension est :

- \textit{algébrique} si tout élément de k est algébrique dans K (i.e est solution d'un polynome à coefficients dans K )
\\ - de \textit{type fini} s'il existe $x_{1},..., x_{n} \in $ K tel que K est engendré par k et $x_{1},..., x_{n}$ ( si $n = 1$ , on dit que l'extension est \textit{simple})

On dit qu'un corps K est algébriquement clos si pour tout polynome non constant à coefficients dans K admet au moins une racine dans K.

Soit k $ \subset $ K deux corps, l'ensembles de points entiers de K sur k sont appellés les \textit{éléments algébriques}, les autres sont dits \textit{transcendants}. Si K est entier sur k, on dit que K est algébrique sur k. La clotute intégrale de k dans K est appelée la cloture algébrique de k dans K. Si k est un corps, il existe une clôture algébrique de k :
c’est une extension k $\subset$ K algébrique et algébriquement close.

On dit qu'un élément de K algébrique sur k est séparable sur k si son polynome minimal a ses  racines distinctes dans une cloture algébrique de K. On dit qu'une extension algébrique K est séparable sur k si tous ces éléments sont séparables sur k.



\section{Ensembles algébriques}

Soit k un corps commutatif, on note $k^{n} = A^{n}$ \textit{l'espace affine} de $k$. C'est l'ensemble des parties de $A^{n}$ définies comme l'ensemble des racines d'un polynome de $k[X_{1},...,X_{n}]$.

Le cercle unité dans $\mathbb{R}^{2}$ est un élément de de $A^{n}$ défini comme $C= \{(x,y) \in \mathbb{R}^{2} , x^{2} + y^{2} - 1 = 0 \} $ 


\centering
\begin{tikzpicture}
\draw (0,0) circle (2cm);
\draw (0,3) -- (0,-3);
\draw (-3,0) -- (3,0);
\end{tikzpicture}


\flushleft \textbf{Autres exemples} :\\
 $Z_{1}=\{(x,y) \in \mathbb{R}^{2},\ y = x^{2}  \}$\\
 $Z_{2} = \{(x,y) \in \mathbb{R} ^{2}, \ x^{3} +  y^{2} = yx^{2} + xy \}  = \{(x,y) \in \mathbb{R} ^{2}, \ x = y\} \cup Z_{1}$  car \ $x^{3} + y^{2} - yx^{2} - xy = (y-x^{2})(y-x) $
\\ $ Z{3} =\{(x,y,z) \in \mathbb{Q}^{3},\ x^{n} + y^{n} = z^{n}\} $   \ \ \ \ \ ($n\geq 1 $)

\underline{Définitions} 
\\
Soit k un corps et n $\geq 1$ et S un partie de $k$[$X_{1},...,X_{n}$]. On appelle \textit{ensemble algébrique défini par S} : \\ \center $V(S) = \left\{  (x_{1}, ... , x_{n}) \in A^{n} , \forall P \in S , \ P(x_{1}, ... , x_{n}) = 0 \ \right\}$
\flushleft

Si R est une partie de $A^{n}$, on définit \textit{l'idéal de R} :
\center
I(R) =$\left\{  P \in k[X_{1},...,X_{n}], \  \forall(x_{1}, ... , x_{n}) \in R \ P(x_{1}, ... , x_{n}) = 0  \right\}$

\underline{Propositions} - Soit S,S' des parties de $k$[$X_{1},...,X_{n}$] et Z,Z' des parties de $A^{n}$. Alors :

 (1) Si $S \subset  S' $ alors $V(S') \subset V(S) $
\\(2) Si $I = (S)$ alors $V(I) =V(S) $
\\(3) $S \subset I(V(S))$
\\ (4) Si $Z \subset Z'$ alors $I(Z') \subset I(Z)$
\\ (5) $I(Z)$ est un idéal de $k[X_{1},...,X_{n}]$
\\ (6) $I(Z \ \cup \ Z') = I(Z)\ \cap \ I(Z')$
\\ (7) $Z\subset V(I(Z)) $ avec égalité si et seulement si $Z$ est un ensemble algébrique

\underline{Preuve} - \\ (1), (4) et (6) proviennent des défintions, pareil pour (3) et pour la première partie de (7).
\\ Montrons (5), soit $P_{1},P_{2} \in I(Z)$ alors sur $Z$ car $P_{1}$ et $P_{2}$ s'annulent donc $P_{1} + P_{2}, \lambda P_{1}$ aussi pour $\lambda \in k$ et donc $\forall Q \in k[X_{1},...,X_{n}], \ P_{1}Q$ s'annule également sur Z alors $I(Z)$ est bien un idéal de $k[X_{1},...,X_{n}]$.
\\ Montons (2), par (1) on $V(I) \subset V(S)$. Soit $P \in I$ alors il existe $ Q_{i} \in S $ et $R_{i} \in k[X_{1},...,X_{n}]$ tel que $P = \sum_{i} P_{i}Q_{i} $. Donc en prenant $ x \in V(S)$, on a $Q_{i}(x) = 0 \  \ \forall i$ donc $ P(x) = 0$, ce qui implique $V(S) \subset V(I)$. D'ou l'égalité.
\\ Montrons la deuxième partie de (7), on suppose que $Z = V(I(Z))$ alors $Z$ est un ensemble algébrique. Si $Z$ est un ensemble algébrique, alors il existe $S \in k[X_{1},...,X_{n}]$ tel que $Z =V(S)$, par définition $S \subset I(Z)$ et par (1) $V(I(Z)) \subset V(S)$ ce qui donne l'inclusion réciproque.

\underline{Corollaire} - Les applications $I$ et $V$ sont des bijections réciproques entre les ensembles algébriques de $ A^{n}$ et les idéaux de $k[X_{1},...,X_{n}]$ de la forme $I(Z)$

\underline{Preuve} - Provient entièrement de la proposition (7). 


\flushleft
\section{Lemme de Zariski et corollaires}

\underline{Corollaire 1} - 
Si $k$ est un corps algébriquement clos alors les idéaux maximaux de $k$[$X_{1},...,X_{n}$] sont du type ($X_{1} - a_{1}, ... ,  X_{n} - a_{n}$) où $a_{1},..., a_{n} \in  k $
\\
\underline{Preuve} : Montrons d'abord que  I = ($X_{1} - a_{1}, ... ,  X_{n} - a_{n}$) pour $a_{1},..., a_{n} \in  k $ est un idéal maximal de $ k$[$X_{1},...,X_{n}$].
\\ On note 
$ \phi & : & k[X_{1},...,X_{n}] & \to & k $ tel que $\phi(P) =P(a_{1},..., a_{n}) $
c'est un morphisme surjectif d'anneaux, on prends P dans Ker($\phi$) et on fait la division euclidienne de P par $ X_{1} - a_{1} $ puis du reste par les autres $X_{i} - a_{i}$ pour $ i \geq 2 $ 
\\
\\
\centering 
$P = (X_{1} - a_{1})Q_{1} + \ ... \ + (X_{n} - a_{n})Q_{n} + P(a_{1},..., a_{n} ) $ avec $Q_{i} \in k[X_{1},...,X_{n}] $
\\
\flushleft Or on sait que $\phi( P) = 0$ donc $ P \in I  $ et Ker($\phi$)\ $\subset$ I , car il est clair que I $\subset$ Ker($\phi$), alors I = Ker($\phi$). En quotientant, on obtient que  $k[X_{1},...,X_{n}]/I \simeq k $ et donc I est un idéal maximal de $ k$[$X_{1},...,X_{n}$].
\\
\\ Montrons désormais le resultat énoncé, soit I un ideal maximal de $k[X_{1},...,X_{n}]$ alors  $k[X_{1},...,X_{n}]/I$ est une k-algèbre de type fini engendrée par $ \pi(X_{1}),...,\pi(X_{n})$ où $ \pi & : & k[X_{1},...,X_{n}] & \to & k[X_{1},...,X_{n}]/I $ \  la projection canonique donc une extension algébrique finie de k d'après le lemme de Zariski, car k est algébriquement clos on a $k[X_{1},...,X_{n}]/I = k $. Finalement, on peut prendre $a_{1},..., a_{n} \in  k $ tel que $X_{i} - a_{i} \in I\ \ , \ \forall i = {1, ... , n} $. On regarde ce que vaut $X_{i} $ mod$[\ I\ ]$.  \\ Donc $(X_{1} - a_{1}, ... ,  X_{n} - a_{n}) \subset I $ comme ($X_{1} - a_{1}, ... ,  X_{n} - a_{n}$) est maximal alors I = ($X_{1} - a_{1}, ... ,  X_{n} - a_{n}$)


\underline{Corollaire 2} - \\
Soit $k$ un corps algébriquemnt clos et I un idéal de $k[X_{1},...,X_{n}] $  qui n'est pas (1) alors V(I) $\neq \emptyset$.

\underline{Preuve} - \\
Soit J un idéal maximal de $k[X_{1},...,X_{n}] $ qui contient I alors il existe $a_{1},..., a_{n} \in  k $ tel que J = ($X_{1} - a_{1}, ... ,  X_{n} - a_{n}$) donc $\forall P \in I$ on a $P(a_{1},..., a_{n}) = 0$ et donc $(a_{1},..., a_{n}) \in V(I) $ et donc $V(I)$ est non vide.

\underline{Corollaire 3}- \\
Soit $k$ un corps algébriquement clos et R un ensemble algébrique de $A^{n}$ défini par l'idéal J. Alors $I(R) = \sqrt{J}$ racine de J.

\underline{Preuve} -\\
- Si $ P \in \sqrt{J} $ alors il existe $m \geq 1$ tel que $ P ^{m} \in J$. Or avec la définition de $R =V(J) = \left\{  (x_{1}, ... , x_{n}) \in A^{n} , \forall P \in J , \ P(x_{1}, ... , x_{n}) = 0 \ \right\}$ donc pour tout $x \in R$, on a $P^{m}(x) = 0$ et donc finalement que $P(x) = 0$ et donc $P \in I(R)$ car $P$ annule tous les éléments de R.

- Si $P \in I(R)$, $k[X_{1},...,X_{n}] $ étant noethérien, il existe $P_{1},..., P_{m} $ dans $k[X_{1},...,X_{n}] $ tels que $J = (P_{1},..., P_{m})$. On remarque que pour l'idéal $ S = (P_{1},..., P_{m},1-TP) $ de $k[X_{1},...,X_{n},T]  ,\ V(S) $ est vide. En effet, si on prend $x \in V(S)$ alors $xP(x) = 1$ et $x \in V(J)$, or $P \in I(V(J))$ donc $P(x) = 0$ et donc $xP(x) = 1$ est absurde.\\ Par le corollaire précédent, on a que $S = (1) =k[X_{1},...,X_{n},T]$. \\
Donc on peut trouver $Q_{j} \in k[X_{1},...,X_{n},T]$ pour $1 \le j \le m$ tels que 

\centering 
$\sum_{j = 1}^m P_{j}Q_{j} \ + (1- TP)Q_{0} = 1$
\flushleft En prenant $T= 1/P$ dans cette équation on obtient

\centering $\sum_{j = 1}^m P_{j}(X_{1},...,X_{n})Q_{j}(X_{1},...,X_{n},1/P)  = 1$
\flushleft Soit $ r\geq 1$ tel que $r$ soit plus grand que le degré des $Q_{j}$ en la variable $T$, de tel sorte que $ \forall \ 1 \le j \le m$ on ait $P^{r} Q_{j}(X_{1},...,X_{n},1/P) \in k[X_{1},...,X_{n}]$. Cela donne donc :

\centering $P^{r} = \sum_{j = 1}^m P_{j}(X_{1},...,X_{n})P^{r}Q_{j}(X_{1},...,X_{n},1/P) $ 
\flushleft Cela implique que $P^{r} \in (P_{1},..., P_{m}) = J$ et finalement $P \in \sqrt{J}$.

Donc $I(V(J)) = \sqrt{J}$

\end{document}
