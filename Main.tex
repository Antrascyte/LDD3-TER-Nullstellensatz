\documentclass[a4paper,10pt]{article}

\usepackage[utf8]{inputenc}
\usepackage[french]{babel}
\usepackage{microtype}
\usepackage{amsmath,amsthm,amsfonts,amssymb}
\usepackage{graphicx}
\usepackage{lmodern}
\usepackage{amsmath,amsfonts,mathrsfs}
\usepackage{gensymb}
\usepackage{lettrine}
\usepackage{calligra}
\usepackage{tikz}
\usepackage{tikz-cd}
\usepackage{tikz-3dplot}
\usepackage{cancel}
\usepackage{boxedminipage}
\usepackage{mathtools}
\usepackage{esint}
\usepackage{stmaryrd}
\usepackage{minted}
\usepackage[european resistor, european voltage, european current]{circuitikz}
\usepackage{bussproofs}
\usepackage{yfonts}
\usepackage{turnstile}
\usepackage{dsfont}
\usepackage{hyperref}
\title{test}
\author{Felice Hopkins}
\date{March 2025}

\begin{document}

\maketitle
\section{Radical}

Soit $A$ un anneau commutatif.


\flushleft\underline{Definition} -Soit $S\subset A $
\\On appelle radical de $S$ l'ensemble :

\centering $ \sqrt{S}:=\{ a\in A \; | \exists n \in \mathbb{N} \; a^n \in I \} 




\flushleft\underline{Propriété:}Soit $I$ un idéal de $A$. Alors $\sqrt{I}$ est un idéal de $A$ contenant $S$.

\underline{Preuve}:

\item[•] $0 \; \in \sqrt{I}$ car $ 0 \in \ I$

Soient $x$ , $y$ $ \in \sqrt{I}$. Il existe $n$ et $k$ $\in \mathbb{N}$ tels que $x^n \in I$ et $y^k \in I $. 
\item[•] $(x-y)^{n+k}= \sum_{j=0}^{n+k} \binom{n+k}{j}x^j(-y)^{n+k-j}$. Chacun des termes de la somme est dans $I$ donc $ (x-y)^{n+k} \; \in I$.

\item[•] $(xy)^{nk}=(x^n)^k((y)^k)^n \; \in I$ 

Conclusion: $\sqrt{I}$ est un idéal de $I$.

\vspace{1\baselineskip}

\\ \underline{Definition}: Un idéal radiciel est un idéal d’un anneau qui est son propre radical.

\vspace{1\baselineskip}

\underline{Propriété}:

Le radical d’un idéal d’un anneau est radiciel

\underline{Preuve}:

Soit $I$ un idéal.
\item[•] , $\sqrt{I} \subset \sqrt{\sqrt{I}} $ est claire
\item[•]Reciproquement, soit $x \in A $ et $n \in \mathbb{N}$ tel que $x^n \in \sqrt{I} $. Soit maintenant $k \in \mathbb{N} $ tel que $(x^n)^k \in I $. On a donc que $x^{nk} \in I}$, donc $x \in \sqrt{I}$. $\sqrt{\sqrt{I}} \subset \sqrt{I} $

Conclusion: $\sqrt{\sqrt{I}} = \sqrt{I} $

\end{document}
